%%This is a very basic article template.
%%There is just one section and two subsections.
\documentclass[a4paper,12pt]{article}
\usepackage[top=2cm, bottom=3cm, left=2.7cm, right=2.7cm]{geometry}
\usepackage{setspace}
\usepackage{graphicx}
\usepackage{fancyhdr}
\usepackage{array,booktabs}
\usepackage{caption}
%\usepackage{tabu}
\pagestyle{fancy}
\lhead{\includegraphics[width=.7cm]{Figures/logo}}
\setlength\headheight{25pt} 

\begin{document}

\title{\vspace{-15mm}
\fontsize{25pt}{10pt}\selectfont
\textbf{
\makebox[0pt][l]{\includegraphics[width=1.5cm]{Figures/logo}}
\hfill
\Huge{Age} \huge{and} \Huge{Time} \hfill\hfill 
} \\ ~ \\~ \\~ \\ \huge{Proposal for Research Group}}
\author{Timothy L. M. Riffe}
\maketitle

\onehalfspacing

\section{Introduction}
I propose to build and lead a research group that will advance fundamental
demography by expanding research based on a seldom-considered realignment of
the age axis in population models --- one which considers remaining time rather
than time lived.
By expanding on the age-classified models presented in my dissertation, this group will contribute models to
mathematical demography and tweak the tools of practical demography to allow for
the discovery and description of empirical regularities within human populations
that have thus far largely gone unnoticed and undescribed.

\subsection{Age axis}
A particular notion of age is pervasive in the social and medical sciences, that
which measures a timespan since birth.
Everything we know about the age patterns of demographic phenomena, about the temporal
unfolding of the lifecourse, and about individual and population ageing itself,
is based on this singular notion of age. 

\begin{figure}[h]
\centering
	\caption{A lifeline, segmented into chronological age (years lived) and
	thanatological age (years left).}
	\label{fig:line}
	\includegraphics[scale=.8]{Figures/LifeLine.pdf}	
\end{figure}

A second notion of age derives from lifespans, the duration between birth and
death, or more generally state entry and exit (see schematic
Figure~\ref{fig:line}).
Age in this case is a distance on the lifeline, measured either from birth (chronological age) or from death
(thanatological age). The first is canonical age, pervasive in this and
other disciplines, and the second is the point of departure for this research
group, and underlies the title ``Age and Time'' (A\&T). This research group
will chart the varied theoretical, methodological, and substantive
implications of thanatological age, thereby contributing to the foundations of
the discipline. The difference between this proposal and the bulk of existing
work on remaining years of life undertaken by demographers
\cite{sanderson2010remeasuring,sanderson2005average,sanderson2007new} is that I
work with distributions of lifelines, whereas most previous work has dealt with
mean remaining lifetime (remaining life expectancy), a summary statistic. This
small difference allows for the estimation of thanatological age structure, and
for the development of a new family of models and results that underlay the
remainder of the proposed A\&T program.

\subsection{Significance}
There is reason to believe that the thanatological age perspective will gain
in importance as the force of mortality falls, which tends to concentrate
most deaths between a narrow range of ages. This tendency increases
certainty with respect to remaining lifespan, which facilitates lifecourse
planning, and which will be ever more relevant as the population pyramid
continues its shift to higher ages. Likely, an increase in conscientious
understanding of remaining lifespan (either from an increase in statistical
awareness or simple peer observation) will influence other demographic and
economic transitions. For example, one could imagine an increase in
post-retirement divorce due to a relaxation of the constraints of poor health
and lifespan. In the recent past such a trend would have been unthinkable.
I argue that this perspective is relevant now, both in terms of perceived and
actual future lifespan, as a sort of intrinsic structure behind demographic
behavior and patterns. These patterns are in need of discovery and explaining
with the new models and measures described in the following.

\section{Focuses}
\subsection{Theoretical work}
Age-classified models, such as those of population renewal and projections, are
redefineable in terms of thanatological age. This was the topic of my
dissertation and subsequent work, in which I developed a model of thanatological
renewal in both continuous and matrix form.\footnote{A draft of a paper
presenting this model is appended with this application.} The formal
relationship between stationary/stable chronological and thanatological age structure that is an artifact of this model was already known \cite{wachter2014essential,vaupel2009life,brouard1989mouvements}, but the model
also yields further insights on the life course\footnote{A paper highlighting
some of these insights is currently under review, appended.} and constitutes new terrain.
Some such results remain identical to those of models structured by chronological age (e.g., the growth rate under certain circumstances), while
others are different (e.g., the hypothetical time and path to stability). I
conjecture that there is a link operation between the chronological age-structured Leslie
matrix and the thanatological age-structured renewal matrix (defined in my
dissertation) that would translate the full suite of canonical age findings into
the thanatological perspective.

Other potent properties of this new renewal model remain unexplored, such
as the meaning of Fischer's reproductive value \cite{fisher1958genetical}, $v(a)$, in
thanatological age, $v^\star (y)$\footnote{Here, $a$ indexes
chronological age and $y$ indexes thanatological age (remaining years).}.
$v^\star (y)$ is an eigenvector property of the thanatological renewal matrix,
but is not yet understood. This is a special case, since $v(a)$ is already a
statement on the future reproduction of age classes, and is forward-looking in a way
analogous to thanatological age. I speculate that the pattern of $v^\star (y)$
says something important about lifespan selection (and possibly other
lifecourse timings), and so may be a breakthrough for evolutionary demography. 

In my dissertation I also found that thanatological aggregates, such as
population structure, are far less variable over time than analogous
chronological age aggregates. Thanatological equivalent measures
typically suggest a rosier picture of current population ageing (and recent
fertility declines) than is typically understood, and thanatological projections in general tend to dampen the demographic consequences of abrupt shifts (for instance, due to the baby
boomers). These findings are in need of deeper understanding, but are pertinent
to contemporary issues of low fertility and population ageing.

\subsection{Methodological work}
Perhaps more relevant to the everyday practice of demography is the expectation
of new measures, decomposition techniques and tools to aid in the measurement
and understanding of population trends that result from the consideration of
thanatological age. A second but equally exciting focus of the A\&T research
group is to develop and disseminate this varied suite of tools. A list of
methods ripe for invention and/or refinement in this direction includes:

\begin{description}
\item[Lifetable Measures]{A large set of lifetable summary measures refer to
the whole lifetable. In each such case, it is possible and
informative to translate the given measure to be conditional on survival, i.e.,
truncating the lifetable as we move up the age scale and recalculating the
measure. In this way, each measure becomes a function of age, and a summary of
the distribution of remaining years of life, which changes over age.\footnote{A
paper on distributional aspects of remaining years of life is currently in
preparation.} The common lifetable column for remaining life expectancy already has this property, but
many more are possible. This work will be of use for late-life decision
analytics (e.g., investments and insurance), but also for the optimization of
age-benchmarked public policies (e.g., mandatory retirement) and
lifetable-based indices of population ageing.}
\item[Decomposition techniques]{Data classified by chronological age is
decomposable into classes of remaining years of life. In the simplest case,
this can yield a picture of the population structured by remaining years of
life,\footnote{A paper on this topic is under review, appended.} but the same basic concept can be expanded to compare the
cause-specific impacts of hypothetical life-saving on population structure or
the cumulative years of life saved, a perspective that complements previous work
on life-saving \cite{vaupel1987repeated,vaupel2008lifesaving}. This is useful
for assessing the raw impacts of causes of death on population stocks.\footnote{A paper on the many temporal perspectives of life-saving is currently in
preparation.} Further work is foreseen in the area of population ageing
indicators based on cross-classified age.}
\end{description}

\subsection{Empirical work}
A third focus of A\&T will be to determine the classes of
life transitions that exhibit strong empirical regularities as a function
of remaining time and to describe these patterns. This work will largely consist
in exploiting pre-existing panel datasets, such as health and retirement
surveys\footnote{A paper exploring thanatological patterns of dependency and
disability in the US Health and Retirement Survey in currently in preparation.}
or youth longitudinal surveys.
Late life transitions are likely to be those best described in terms of
remaining years of life, but this temporal perspective generalizes to other
terminal events, such as marriage (divorce), employment (retirement), and health/morbidity transitions at various points in
the lifecourse, which makes the thanatological age perspective pertinent to
the whole lifespan and multiple aspects of population research. 

Full lifecourse age patterns, such as those of
thanatological reproductive value, and fertility itself, are also in need of empirical description, most likely on the basis of long-running population registers (such as the Scandinavian registers) or linked historical
data sources (such as the UC Croatia project data).
An ambitious, curiosity-driven survey of these areas is called for.
Such empirical work is basic scientific discovery, and I consider the field wide
open at this point. Measurement challenges faced in these efforts will lead
to further methodological developments, for instance in the estimation of the
distribution of remaining years of good health. The importance of
such findings in inestimable, but it must be recalled that everything we know
about age was sparked by first observing a regular age pattern, such as Gompertz
mortality, and later searching for explanations and mechanisms.

\section{Conclusion}
The proposed A\&T program is unified in its basis on thanatological age
structure. The level and breadth of impact foreseen, both on demographic
theory and praxis, as well as relevance to current population issues, justifies
institutional investment. The varied and deep subareas within this program
justify a research team rather than a personal grant. The MPIDR is the most suitable
institution to host this project in the world, as existing labs and projects
complement the work presented here, and would form the basis of strong internal collaboration.
My creative thinking, scholarly potential, ambition, international outlook,
personal research network, and ability to forge new working relationships
justify me as the leader of the Age and Time group.


\pagebreak
\section{Work plan}
The group focuses outlined above are diverse and
ambitious, and so the team required to execute these must also be diverse and
ambitious. I will find and gather a small group of enterprising, curiosity-driven and complementary individuals to undertake this work and contribute ideas and vision to the project. The assimilation of PhD students will be staggered over the first three years of the project, and a post doc will be sought out for early incorporation.

Work on all focus areas will be simultaneous throughout the project, but group
emphasis will shift over the five years of the project. Work in years one
through three will emphasize theoretical and mathematical foundations, as well
as novel methodological work. Project effort will be invested in the second year
to collect and catalogue data sources that allow for direct estimation of
demographic phenomena as a function of remaining years. Years four and five will shift emphasis to the discovery and description of thanatological age patterns of varied demographic phenomena in different contexts.

\begin{table}[h]
\centering
\caption*{Calendar of work emphasis}
   \begin{tabular}{rll}
 &
\raisebox{-.25\height}{\includegraphics[scale=1]{Figures/Gantt/0_Years}}
 & Project Years\\
Math \& Theory &
\raisebox{-.25\height}{\includegraphics[scale=1]{Figures/Gantt/1_Math}}&
\\
Methods &
\raisebox{-.25\height}{\includegraphics[scale=1]{Figures/Gantt/1_Methods}}&
\\
Data & \raisebox{-.25\height}{\includegraphics[scale=1]{Figures/Gantt/2_Data}}&
\\
Empirics &
\raisebox{-.25\height}{\includegraphics[scale=1]{Figures/Gantt/3_Empirical}}& \\
   \end{tabular}
\end{table}

Finally, all published work, derived results and code pertaining to
methodological advancements will be strictly open access.

\pagebreak
\bibliographystyle{unsrt}
  \bibliography{references}  

\end{document}
