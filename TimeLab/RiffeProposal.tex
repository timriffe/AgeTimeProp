%%This is a very basic article template.
%%There is just one section and two subsections.
\documentclass[a4paper,12pt]{article}
\usepackage[top=2cm, bottom=3cm, left=3cm, right=3cm]{geometry}
\usepackage{setspace}
\usepackage{graphicx}
\usepackage{fancyhdr}
\usepackage{array,booktabs}
\usepackage{caption}
%\usepackage{tabu}
\pagestyle{fancy}
\lhead{\includegraphics[width=1cm]{Figures/logo}}
\setlength\headheight{40pt} 

\begin{document}

\title{\vspace{-15mm}
\fontsize{25pt}{10pt}\selectfont
\textbf{
\makebox[0pt][l]{\includegraphics[width=1.5cm]{Figures/logo}}
\hfill
\Huge{Age} \huge{and} \Huge{Time} \hfill\hfill 
} \\ ~ \\~ \\~ \\ \huge{Proposal for Research Group}}
\author{Timothy L. M. Riffe}
\maketitle

\onehalfspacing

%a. The basic idea of the proposal is great and innovative. The proposal would be stronger if you (a)
% relate the proposal of thanatological age to work by others who focus on remaining lifetime (see 
% review paper by Sanderson and Scherbov), (b) stress that your focus is on age-classified models 
% in terms of thanatological age, and (c ) say more about what you did and found in your dissertation
% (e.g. eTFR). Your dissertation is strong basis. The critical question that a member of the 
% selection committee asks is: is the research innovative and what will we learn? What are we 
% missing if the research is not funded? What will we know what we cannot know otherwise?
%b. If you relate thanatological age to the reproductive value of Fisher (as you did in your 
%dissertation), that would make the proposal stronger because the reproductive value is well-known 
%in demography and is a major concept in genetics and evolution. I would not stress the reproductive 
%value 'flipped', but the fact that the r.v. measures remaining number of children (and hence, r.v.
% is a thanatological measure). If the fertility rate is replaced by one, the reproductive value 
% measures the discounted life expectancy (in the stationary population: the life expectancy). If 
% you extend the view that the r.v. is a thanatological measure and consider the population growth 
% matrix with fertility replaced by other useful rates of events in the life course that lead to
% some type of renewal, than the right eigenvector of that matrix relates to chronological age 
% and the left eigenvector to thanatological age. That would be a breakthrough if that is correct.
% I suggest you add some of this speculation (but with more detail). 
%c. Figure 4.16 of your dissertation. It seems to me that the r.v. should slightly increase between
% ages 0 and 15 due to mortality. The increase may not be visible in the figure. Note that the age
% curve of r.v. has a regular pattern.  Since r.v. is closely related to thanatological age, you 
% may consider mentioning that in section 2.3 when you talk about regular age patterns. 
%d. Figure 1. You refer to a and y, but they are not shown in the figure. 
%e. I suggest not to talk about a half-draft. Just say that a paper is in preparation. 
%f. Since you refer to effects of life-saving, you should probably know (or refer to) Jim 
%Vaupel's paper on the impact of saving a life on life expectancy. 


\section{Introduction}
I propose to build and lead a research group that will advance fundamental
demography by expanding research based on a seldom-considered realignment of
the age axis in population models. This group will advance the tools of
mathematical demography and tweak the tools of practical demography to allow for
the discovery and description of empirical regularities within human populations that have thus far largely gone unnoticed and undescribed.

\subsection*{Age axis}
A particular notion of age is pervasive in the social and medical sciences, that
which measures a timespan since birth.
Everything we know about the age patterns of demographic phenomena, about the temporal
unfolding of the lifecourse, and about individual and population ageing itself,
is based on this singular notion of age. 

\begin{figure}[h]
\centering
	\caption{A lifeline, where chronological age (years lived) is indexed by $a$
	and thanatological age (years left) is indexed by $y$.}
	\label{fig:line}
	\includegraphics[scale=.8]{Figures/LifeLine.pdf}	
\end{figure}

A second notion of age derives from lifespans, the duration between birth and
death, or more generally state entry and exit (see schematic
Figure~\ref{fig:line}).
Age in this case is a distance on the lifeline, measured either from birth (chronological age) or from death
(thanatological age). The first is canonical age, pervasive in this and
other disciplines, and the second is one point of departure for this research
group, and underlies the title ``Age and Time'' (A\&T). This research group
will chart the varied theoretical, methodological, and substantive
implications of thanatological age, thereby contributing to the foundations of the discipline.

\section{Focuses}
\subsection{Theoretical work}
Age-classified models, such as those of population renewal or projections, are
redefineable in terms of thanatological age. This was the topic of my
dissertation and some subsequent work (either under review or in preparation).
Some general results remain identical to models structured by chronological age (e.g., the growth rate under certain
circumstances), while others are different (e.g., the hypothetical time to
stability). One focus of the A\&T research group will be to
expand and develop the continuous and discrete mathematics of these models to explain such results and therein round-out the existing foundation of population theory.

\subsection{Methodological work}
Perhaps more relevant to the everyday practice of demography is the expectation
of new measures, decomposition techniques and tools to aid in the measurement
and understanding of population trends that result from the consideration of
thanatological age. A second but equally exciting focus of the A\&T research
group is to develop and disseminate this varied suite of tools. A list of
methods ripe for invention and/or refinement in this direction includes:

\begin{description}
\item[Lifetable Measures]{A large set of lifetable summary measures refer to
the whole lifetable. In each such case, it is possible and
informative to translate the given measure to be conditional on survival, i.e.,
truncating the lifetable as we move up the age scale and recalculating the
measure. In this way, each measure becomes a function of age, and a summary of
the distribution of remaining years of life, which changes over age. The common
lifetable column for remaining life expectancy already has this property, but
many more are possible. A paper on this topic is currently in preparation.
This work will be of use for late-life decision analytics (investments and
insurance), but also for the optimization of age-benchmarked public policies
and lifetable-based indices of population ageing.}
\item[Decomposition techniques]{Data classified by chronological age is
decomposable into classes of remaining years of life. In the simplest case,
this can yield a picture of the population structured by remaining years of life, but
the same basic concept can be expanded to compare the cause-specific impacts of
life-saving on population structure or the cumulative
years of life saved, a perspective that complements previous work on life-saving
\cite{vaupel1987repeated,vaupel2008lifesaving}. This is useful for
assessing the raw impacts of causes of death on population stocks. A paper on the many temporal perspectives of life-saving is currently in prepration.}
\end{description}

\subsection{Empirical work}
A third focus of A\&T will be to determine the classes of
life transitions that exhibit strong empirical regularities as a function
of remaining time and to describe these patterns. This work will largely consist
in exploiting pre-existing panel datasets. Late life transitions are likely to
be those best described in terms of remaining years of life, but this
perspective generalizes to other terminal events, such as marriage (divorce) or
employment (retirement), and so can enrich various areas of population
research. An ambitious, curiosity-driven survey of these areas in called for.
Such empirical work is basic scientific discovery, and I consider the field wide
open at this point. The importance of such findings in inestimable, but it must
be recalled that everything we know about age was sparked by first observing a
regular age pattern, such as Gompertz mortality, and later
searching for explanations and mechanisms.

\pagebreak
\section{Work plan}
The group focuses outlined above are diverse and
ambitious, and so the team required to execute these must also be diverse and
ambitious. I will find and gather a small group of enterprising, curiosity-driven and complementary individuals to undertake this work and contribute ideas and vision to the project. The assimilation of PhD students will be staggered over the first three years of the project, and a post doc will be sought out for early incorporation.

Work on all focus areas will be simultaneous throughout the project, but group
emphasis will shift over the five years of the project. Work in years one
through three will emphasize theoretical and mathematical foundations, as well
as novel methodological work. Project effort will be invested in the second year
to collect and catalogue data sources that allow for direct estimation of
demographic phenomena as a function of remaining years. Years four and five will shift emphasis to the discovery and description of thanatological age patterns of varied demographic phenomena in different contexts.

\begin{table}[h]
\centering
\caption*{Calendar of work emphasis}
   \begin{tabular}{rll}
 &
\raisebox{-.25\height}{\includegraphics[scale=1]{Figures/Gantt/0_Years}}
 & Project Years\\
Math \& Theory &
\raisebox{-.25\height}{\includegraphics[scale=1]{Figures/Gantt/1_Math}}&
\\
Methods &
\raisebox{-.25\height}{\includegraphics[scale=1]{Figures/Gantt/1_Methods}}&
\\
Data & \raisebox{-.25\height}{\includegraphics[scale=1]{Figures/Gantt/2_Data}}&
\\
Empirics &
\raisebox{-.25\height}{\includegraphics[scale=1]{Figures/Gantt/3_Empirical}}& \\
   \end{tabular}
\end{table}

Finally, all published work, derived results and code pertaining to
methodological advancements will be strictly open access.

---------------------------------------
\bibliographystyle{plain}
  \bibliography{references}  

\end{document}
