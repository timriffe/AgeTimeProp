%%This is a very basic article template.
%%There is just one section and two subsections.
\documentclass[a4paper,12pt]{article}
\usepackage[top=2cm, bottom=3cm, left=3cm, right=3cm]{geometry}
\usepackage{setspace}
\usepackage{graphicx}
\usepackage{fancyhdr}
\pagestyle{fancy}
\lhead{\includegraphics[width=1cm]{Figures/logo}}
\setlength\headheight{40pt} 

\begin{document}

\title{\vspace{-15mm}
\fontsize{25pt}{10pt}\selectfont
\textbf{
\makebox[0pt][l]{\includegraphics[width=1.5cm]{Figures/logo}}
\hfill
\Huge{Time} \huge{and} \Huge{Population} \hfill\hfill 
} \\ ~ \\~ \\~ \\ \huge{Proposal for Research Group}}
\author{Timothy L. M. Riffe}
\maketitle

\onehalfspacing

\section{Background}

I propose to build and lead a research group that will do fundamental
demography. This group will advance the tools of mathematical demography and
tweak the tools of practical demography to allow for the discovery and description of empirical regularities
within human populations that have thus far largely gone unnoticed and
undescribed. Such progress is possible due to a seldom-considered realignment of
the age axis in population models. 

\subsection*{Age axis}
A particular notion of age is pervasive in the social and medical sciences, that
which measures a timespan since birth.
Everything we know about the age patterns of demographic phenomena, about the temporal
unfolding of the lifecourse, and about individual and population ageing itself,
is based on this singular notion of age. This age-perspective is also used in
survival analysis in general, even though the durations under study may not be
strictly age, and it therefore conditions findings from this
family of methods.

\begin{figure}[h]
\centering
	\caption{A lifeline, where chronological age (years lived) is indexed by $a$
	and thanatological age (years left) is indexed by $y$.}
	\label{fig:line}
	\includegraphics[scale=.8]{Figures/LifeLine.pdf}	
\end{figure}

A second notion of age derives from lifespans, the duration between birth and
death, or more generally state entry and exit (see schematic
Figure~\ref{fig:line}).
Age in this case is a distance on the lifeline, measured either from birth (chronological age) or from death
(thanatological age). The first is canonical age, pervasive in this and
other disciplines, and the second is one point of departure for this research
group, and underlies the title ``Time and Population''. This research group
will chart the varied theoretical, methodological, and substantive
implications of thanatological age, thereby contributing to the foundations of the discipline.

\subsection*{Theoretical work}
Age-classified models, such as those of population renewal or projections, are
redefineable in terms of thanatological age. This was the topic of my
dissertation and some subsequent work (either under review or in preparation).
Some general results remain identical to models structured by chronological age (e.g., the growth rate under certain
circumstances), while others are different (e.g., the hypothetical time to stability). One focus of the Time research group will be to expand and develop the continuous and discrete mathematics of
these models to explain such results and therein round-out the existing
foundation of population theory.
 
\subsection*{Methodological work}
Perhaps more relevant to the everyday practice of demography is the expectation
of new measures, decomposition techniques and tools to aid in the measurement
and understanding of population trends that result from the consideration of
thanatological age. A second but equally exciting focus of the Time research
group is to develop and disseminate this varied suite of tools. A list of
methods ripe for invention and/or refinement in this direction includes:

\begin{description}
\item[Lifetable Measures]{A large set of lifetable summary measures refer to
the whole lifetable. In each such case, it is possible and
informative to translate the given measure to be conditional on survival, i.e.,
truncating the lifetable as we move up the age scale and recalculating the
measure. In this way, each measure becomes a function of age, and a summary of
the distribution of remaining years of life, which changes over age. The common
lifetable column for remaining life expectancy already has this property, but many more are possible. A half-draft
starting on this topic is currently in preparation with two coauthors.
This work will be of use for late-life decision analytics (investments and insurance), but also for the optimization of age-benchmarked public policies.}
\item[Decomposition techniques]{Data classified by chronological age is
decomposable into classes of remaining years of life. In the simplest case,
this can yield a picture of the population structured by remaining years of life, but
the same basic concept can be expanded to compare the cause-specific impacts of
life-saving on population structure or the cumulative
years of life saved. This is useful for assessing the raw impacts of causes of
death on population stocks. A half-draft on the many temporal perspectives of life-saving is currently in prepration with a coauthor}
\end{description}

Other methodological innovations will 

\subsection{Empirical work}




\section{Work plan}


\end{document}
